\documentclass{cascadilla-xelatex-biblatex}

\title{Using \texttt{cascadilla-xelatex-biblatex.cls}\thanks{András Bárány, Leiden University Centre for Linguistics,
\url{a.barany@hum.leidenuniv.nl}. This class is heavily influenced by Max
Bane's original \texttt{cascadilla.cls} class.}}

\author{András Bárány}

\begin{document}

\maketitle

\section{Section heading}

\imsubsection{Immediate subsection heading}

Some text.

\subsection{Subsection following text}

\imsubsubsection{Immediate subsubsection heading}

Some more text.

\section{Citing other work}

\texttt{cascadilla-xelatex-biblatex.cls} uses \texttt{biblatex} and
\texttt{biber}. The class uses the \texttt{biblatex} implementation of the
Unified Style Sheet for
Linguistics.\footnote{\url{https://github.com/semprag/biblatex-sp-unified/}}
Note that the bibliography styles \textbf{do not come with the class}:
please download them from \url{https://github.com/semprag/biblatex-sp-unified/}
and place the \texttt{.cbx} and \texttt{.bbx} files in the folder of your
document (or the relevant folder on your system).

Here's an example in-text citation: \textcite{EKiss2008}. By default, the
bibliography file is assumed to have the same name as the \texttt{.tex} file,
in this case \texttt{example.bib}. To change this, use
\verb+\addbibresource{...}+ to specify the location of your bibliography in
your preamble.

\section{Comparison to \texttt{cascadilla.cls}}

This class is heavily influenced by Max Bane's
\texttt{cascadilla.cls}\footnote{\url{https://github.com/maxbane/cascadilla.cls/}},
but it uses \XeLaTeX{} and \texttt{biblatex} instead of (pdf)\LaTeX{} and
\texttt{natbib} and is based on an arguably more recent version of the
Cascadilla Proceedings Project style
sheet.\footnote{\url{http://www.lingref.com/cpp/authors/style.html}}

By default, the class uses the TeX Gyre Termes typeface instead of Times (or
Times New Roman) as you might not have Times (or Times New Roman) installed on
your system. If you want to use Times New Roman instead, you can use the option
\texttt{timesnewroman} when specifying the document class. You can also use
specify your own Times-like font in the preamble using the usual
\texttt{fontspec} commands.

\section{Using this class for Cascadilla proceedings}

I started modifying Max Bane's original class for my own use. There's no
guarantee or proof (yet) that a file created using the current class will be
accepted for actual proceedings published by the Cascadilla Proceedings
Project, but that's the aim.

\printbibliography

\end{document}

