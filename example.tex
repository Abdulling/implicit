\documentclass{cascadilla-xelatex-biblatex}

\ifthenelse{\boolean{usegb4e}}{}{%
    \usepackage{xcolor}
    \usepackage{mdframed}
    \surroundwithmdframed[backgroundcolor=gray!10!white]{verbatim}
}

\title{Using \texttt{cascadilla-xelatex-biblatex.cls}\thanks{András Bárány,
        Bielefeld University,
\url{andras.barany@uni-bielefeld.de}. This class is heavily influenced by Max
Bane's original \texttt{cascadilla.cls} class.}}

\author{András Bárány}

\addbibresource{example.bib}

\begin{document}

\maketitle

\section{Section heading}

\imsubsection{Immediate subsection heading}

Some text.

\subsection{Subsection following text}

\imsubsubsection{Immediate subsubsection heading}

Some more text.

\section{Basics and class options}

This class is heavily influenced by Max Bane's
\texttt{cascadilla.cls}\footnote{\url{https://github.com/maxbane/cascadilla.cls/}},
but it can be used with \hologo{XeLaTeX} instead of (pdf)\hologo{LaTeX}.

A bigger change is that the class obligatorily uses
\texttt{biblatex}/\hologo{biber} instead of \texttt{natbib}. \texttt{biblatex}
and \hologo{biber} are being actively developed. The main thing that changes
for users is that to compile the bibliography, one has to use the
\texttt{biber} command rather than the \texttt{bibtex} command. You can use the
same citation commands you used with \texttt{natbib}/\texttt{bibtex}. See
also~\Cref{sec:bib}.

Finally, the class is based on a more recent version of the Cascadilla
Proceedings Project style
sheet.\footnote{\url{http://www.lingref.com/cpp/authors/style.html}}

\subsection{Specifying the title and author(s)}

You can use the usual \verb+\title{...}+ and \verb+\author{...}+ commands to
specify the title and author(s) of your paper. The starred footnote seen at the
bottom of this page is added using the command \verb+\thanks{...}+. For the
present document, this looks as follows:

\begin{verbatim}
\title{Using \texttt{cascadilla-xelatex-biblatex.cls}%
    \thanks{András Bárány, Bielefeld University,
    \url{andras.barany@uni-bielefeld.de}. This class is heavily
    influenced by Max Bane's original \texttt{cascadilla.cls} class.}}

\author{András Bárány}
\end{verbatim}

Title and author(s) are added to the document using \verb+\maketitle+, as
usual.

\subsection{Options}
\imsubsubsection{XITS vs. Times}

When using \hologo{XeLaTeX}, the class uses the XITS fonts by default, instead
of Times (or Times New Roman) as you might not have Times (or Times New Roman)
installed on your system (I don't). If you want to use Times New Roman instead,
you can use the option \texttt{times} when specifying the document class. You
can also use specify your own Times-like font in the preamble using the usual
\texttt{fontspec} commands.

\subsubsection{A4 vs.\ letter}

You can choose between A4 and letter paper formats. By default, A4 is used, you
can change this by adding the \texttt{letter} option when loading the class.

\subsubsection{Packages for linguistic examples}

I use \texttt{expex}\footnote{\url{https://ctan.org/pkg/expex}} but you can use
\texttt{gb4e}\footnote{\url{https://www.ctan.org/pkg/gb4e}} or
\texttt{linguex}\footnote{\url{https://www.ctan.org/pkg/linguex}} by specifying
the relevant options.

\subsection{Specifying options}\label{sub:ex}

To load the class with particular options, edit the class call. The following
lines would load the class with the \texttt{letter} and \texttt{gb4e} options.

\begin{verbatim}
\documentclass[letter,gb4e]{cascadilla-xelatex-biblatex}
\end{verbatim}

\section{Citing other work}\label{sec:bib}

\texttt{cascadilla-xelatex-biblatex.cls} uses \texttt{biblatex} and
\texttt{biber}. The class uses the \texttt{biblatex} implementation of the
Unified Style Sheet for
Linguistics.\footnote{\url{https://github.com/semprag/biblatex-sp-unified/}}
Note that the bibliography styles \textbf{do not come with the class}: please
download them from \url{https://github.com/semprag/biblatex-sp-unified/} and
place the \texttt{.cbx} and \texttt{.bbx} files in the folder of your document
(or the relevant folder on your system).

Here are examples of in-text citations: \textcite{Yuan2021} and
\textcite{EKiss2008}. The bibliography file is specified by the
\verb+\addbibresource{...}+ command in the preamble of a document. In the
current file, this looks as follows:

\begin{verbatim}
\addbibresource{example.bib}
\end{verbatim}

\section{Using this class for Cascadilla proceedings}

I started modifying Max Bane's original class for my own use. While there is no
guarantee that a file created using the current class will be \emph{definitely}
be accepted for actual proceedings published by the Cascadilla Proceedings
Project, it has worked for me at least once. In addition, the class has now (as
of January 2021) been checked by editors at Cascadilla Proceedings Press and is
linked to on their website.

\newrefcontext[sorting=nyt]
\printbibliography

\end{document}

